\documentclass{article}

\begin{document}

\title{Problem #2\\ Measurement instruments}

You may be familiar with the problem of trying to build something and
wondering why the pieces just don't line up. Would switching measuring
tapes, or making sure you always use the same measuring tape make a
difference?

Many of you may have also noticed that clocks tend to drift apart over
time, even if you reset them periodically. How will you know when you
are late to class? To appointments elsewhere?

In the context of a laboratory class, similar problems may arise. How
much will variations in measurements effect your results in lab
problems this semester? This is something you will be able to explore
in this problem.

\section{Equipment}
In this problem, a number of meter sticks, rulers, and metal tracks
with measuring tapes on them will be available in the classroom. You
will also have access to a balance scale, a toy car, and a stopwatch.


\section{Prediction}

What element of the procedure chosen do you think effects length,
time, or mass measurements the most? How can you compare two
measurements? How does a toy car's travel time change with distance?

\section{Warm-Up}

Read appendix B and appendix C in the lab manual.

\begin{itemize}
\item If you measure a length three times and obtain 40 cm each time,
  what is the precision?
\item If you measured a distance across the room and obtained the
  following data set, what is the uncertainty? 5.30 m, 5.20 m, 5.33 m,
  5.25 m
\item If you time a toy car over one distance in one trial, how should
  changing the distance effect the time you measure?
\end{itemize}

\section{Exploration}

Have each person in your group measure the length of a toy car. How
precisely did you measure it?

Try varying your measurement procedure. Does it matter which part of
the ruler or meter stick you use? Does it matter if you use different measuring devices? Can you think of other ways to vary your procedure? 

Measure the mass of your toy car. Does it matter whether or not you
zero the balance scale first? Does it matter if you push it, or look
from a different angle? 

Time your toy car traveling across a table. Is this easier for longer
or shorter distances? How much do you have to change the distance by
in order for the new time it takes to be noticeable compared to the
variations of using the stopwatch? Record your observations in your
lab notebook.

\section{Measurement}

Measure the length of your toy car again. Switch toy cars with another
group, and measure the length of the new toy car. What is the
uncertainty of these measurements? Record your data and your procedure
in your lab notebooks.

Measure the mass of the new toy car. How close is it to the mass you
previously measured?

Measure the time it takes for the toy car to travel across each of
several different distances. Record the distances, times, and
estimates for the amount of error in your lab notebook.

\section{Analysis}

Did the two toy cars you used have the same length? The same mass?

Plot the times and distances using error bars. Was it possible to tell
from this plot what the overall trend was? If so, what do you think it
is, and how can you test this using your data?

\section{Conclusion}

What factors lead to the greatest variations in length, mass, and time
measurements? In an experiment involving all three variables, how can
you tell which uncertainty matters most? If you increased or decreased
the distance a toy car traveled, how did it effect its travel time?
Are you able to explain any trends in your data, or any of the effects
you saw?
